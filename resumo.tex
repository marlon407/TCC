% RESUMOS
% resumo em português
\setlength{\absparsep}{18pt} % ajusta o espaçamento dos parágrafos do resumo
\begin{resumo}

    Com a ascensão de dispositivos móveis que reivindicam uma parcela cada vez maior do tráfego de internet, otimizar o desempenho da busca de dados torna-se mais importante. A arquitetura REST tem sido durante muito tempo a solução mais comum ao desenvolver APIs Web, mas o GraphQL tem se tornado, em tempos recentes, uma alternativa interessante ao REST. 
    
    O presente trabalho aborda as principais técnicas e conhecimentos sobre comunicação entre aplicações Web. Também é realizado neste trabalho um experimento, tendo como principal parâmetro de comparação o desempenho de APIs implementando REST e GraphQL quando realizado buscas de objetos aninhados. Protótipos de cada API foram implementados e usados para realizar medições do desempenho de cada técnica. O GraphQL apresentou um melhor desempenho em praticamente todos os cenários executados no experimento.

\textbf{Palavras-chave}: Comunicação. REST. GraphQL. Desempenho.

\end{resumo}

% resumo em inglês
\begin{resumo}[Abstract]
 
   This is the english abstract.

   \vspace{\onelineskip}
 
   \noindent 
   \textbf{Keywords}: latex. abntex. text editoration.
 
\end{resumo}
