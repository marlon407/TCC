% RESUMOS
% resumo em português
\setlength{\absparsep}{18pt} % ajusta o espaçamento dos parágrafos do resumo
\begin{resumo}

    Com a ascensão de dispositivos móveis que reivindicam uma parcela cada vez maior do tráfego de internet, otimizar o desempenho da busca de dados torna-se muito importante. A arquitetura REST tem sido, durante muito tempo, a solução mais comum ao desenvolver APIs Web, mas o GraphQL tem se tornado, em tempos recentes, uma alternativa atrativa. O presente trabalho aborda as principais técnicas e conhecimentos sobre comunicação entre aplicações Web. Também é realizado um experimento, tendo como principal parâmetro de comparação o desempenho de APIs implementando REST e GraphQL quando realizado buscas de objetos aninhados. Protótipos de cada API foram implementados e usados para realizar medições do desempenho de cada técnica. O GraphQL apresentou um melhor desempenho em praticamente todos os cenários executados no experimento.

\textbf{Palavras-chave}: Comunicação. REST. GraphQL. Desempenho.

\end{resumo}

% resumo em inglês
\begin{resumo}[Abstract]
 
    With the rise of mobile devices claiming a growing share of internet traffic, optimizing data search performance becomes very important. The REST architecture has long been the most common solution when developing Web APIs, but GraphQL has become, recently, an attractive alternative. This paper discusses the key techniques and knowledge about data communication between Web applications. An experiment is also performed, having as main parameter of benchmarking the performance of APIs implementing REST and GraphQL when performing requests to nested objects. Prototypes of each API were implemented and used to carry out measurements of each technique performance. GraphQL performed better in virtually all scenarios running in the experiment.

   \vspace{\onelineskip}
 
   \noindent 
   \textbf{Keywords}: Communication. REST. GraphQL Performance.
 
\end{resumo}
