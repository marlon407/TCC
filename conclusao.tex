% ---
% Conclusão
% ---
\chapter{Conclusão}
% ---

Com cada vez mais dispositivos interconectados, a eficiência na comunicação entre aplicações é um tema que está em constante debate e evolução. Recursos computacionais como memórias e processadores, embora gradativamente mais acessíveis, ainda demandam preocupação em relação ao seu uso sustentável, principalmente levando em consideração serviços \textit{on demand} em nuvem.

Este trabalho apresentou conceitos e modelos de comunicação entre aplicações, abrangendo algumas de suas vantagens e desvantagens. Além disso, foram levantadas quatro hipóteses referentes ao desempenho de APIs implementando protótipos de dois dos modelos apresentados: REST e GraphQL. Com a utilização dos modelos, um experimento foi realizado para compará-los. A realização do experimento mensurou o desempenho das APIs em termos de tempo de resposta, quantidade de banda utilizada e consumo de recursos computacionais.

A primeira hipótese sugeria que o tamanho da resposta seria menor para as consultas à API GraphQL. Os resultados do experimento consolidaram esta hipótese como verdadeira, mostrando que a API GraphQL foi em todos os cenários propostos mais eficiente do que a API REST, sendo que em alguns cenários a diferença entre as duas APIs foi de quase 80\%. Este resultado já era esperado, pois como princípio, APIs REST expõem recursos, e a consulta destes recursos sempre retornam a representação completa dos objetos. Por outro lado, APIs GraphQL respondem apenas os campos solicitados pela consulta, resultando em respostas mais concisas.

A segunda hipótese declarava que o tempo de resposta também seria menor para as consultas à API GraphQL. Novamente os resultados confirmaram esta hipótese, e em nenhum dos cenários a API REST se mostrou mais eficiente do que a API GraphQL neste quesito. Em um dos cenários, a API GraphQL respondeu a consulta na metade do tempo necessário para a API REST, sendo que em nenhum cenário a diferença foi menor do que 15\%. Os resultados da primeira hipótese podem ajudar um pouco a explicar por que os tempos de resposta da API GraphQL são menores do que os tempos da API REST. Uma vez que é preciso trafegar menos informações pela rede, as consultas são naturalmente respondidas mais rapidamente. Entretanto, outro fator que influencia o tempo de resposta, é quanto tempo a consulta leva para ser processada.

A terceira hipótese afirmava que o tempo de utilização da CPU para as requisições REST seria maior do que o tempo para as requisições GraphQL. Os resultados mostraram que embora em um dos cenários a utilização da CPU foi praticamente igual entre as duas APIs, a medida que é necessário realizar consultas em objetos maiores, o tempo de utilização da CPU da API REST aumenta signitivamente, alcançando uma diferença de mais de 40\% em relação a API GraphQL.

A última hipótese alegava que o consumo de memória da API REST seria menor do que o consumo de memória da API GraphQL. Outra vez, os resultados refutaram esta hipótese. Foi ao avaliar este quesito que o único cenário em que a API REST teve uma eficiência maior do que a API GraphQL foi encontrado. Entretanto, enquanto a diferença de desempenho entre as API foi de apenas 12\% no cenário favorável ao REST, no cenário em que o GraphQL foi mais eficiente, a diferença foi de 30\%. 

Ao fim do trabalho conclui-se que ambas as tecnologias oferecem uma solução prática e eficiente para o mesmo problema, entretanto o GraphQL apresenta-se como uma ótima alternativa como mecanismo de comunicação entre aplicações. A facilidade de uso, menor consumo de recursos computacionais e de banda contribuem para que o GraphQL seja uma ferramenta muito utilizada futuramente pelas organizações e desenvolvedores. 

Por outro lado, os critérios de desempenho podem não ser os únicos benefícios trazidos com a utilização do GraphQL. Posto que, os desenvolvedores das aplicações clientes são encarregados de definir cada requisição, eles podem ser bastante seletivos em relação os dados trafegados. Desde modo, o GraphQL ajuda a maximizar a produtividade do desenvolvimento de aplicações, resultando em uma melhor \textit{Developer Experience} (DX), uma disciplina que vem ganhando cada vez mais dedicação e importância dentro das empresas.

Trabalhos futuros poderiam abordar como a utilização de arquiteturas baseadas em JSON/Graphs (GraphQL, Falcor), podem otimizar o desenvolvimento de \textit{softwares}, não apenas em uma maneira quantitativa, mas também de maneiras qualitativas, levando em consideração a \textit{Developer Experience}. Outros trabalhos também poderiam abordar as vantagens de ter uma comunicação fortemente tipada, como ocorre com o GraphQL. Ou ainda, se a teoria das categorias pode ser aplicada através do GraphQL.
