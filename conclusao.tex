% ---
% Conclusão
% ---
\chapter{Conclusão}
% ---

Com cada vez mais dispositivos conectados, a eficiencia na comunicação entre aplicações é um tema que esta em constante debate e evolução. Recursos computacionais como memórias e processadores, embora gradadivamente mais acessiveis, ainda demandam preocupacão em relação ao seu uso sustentável, principalmente levando em consideração serviços na nuvem \textit{on demand}.

Este trabalho apresentou conceitos e modelos de comunicação entre aplicações, abragendo algumas de suas vantagens e desvantagens. Além disso, foram levantadas quatro hipóteses referentes ao desempenho de APIs implementando dois dos modelos apresentados: REST e GraphQL. Com a utilização dos modelos, um experimento foi realizado para comparálos. A realização do experimento mensurou o 
desempenho das APIs em termos de tempo de resposta, quantidade de banda utilizada e consumo de recursos computacionais.

A primeira hipótese sugeria que o tamanho da resposta seria menor para as consultas à API GraphQL. Os resultados do experimento consolidaram esta hipótese como verdadeira, mostrando que a API GraphQL foi em todos os cenários propostos mais eficiente do que a API REST, sendo que em alguns cenários a diferença entre as duas APIs foi de quase 80\%. Este resultado já era esperado, pois como princípio, a API REST expõe recursos, e a consulta destes recursos sempre retornam a representação dos objetos completos. Por outro lado, a API GraphQL responde apenas os campos solicitados pela consulta, resultando em resposta mais concisas.

A segunda hipótese dizia que o tempo de resposta também seria menor para as consultas à API GraphQL. Novamente os resultados confirmaram esta hipótese, e em nenhum dos cenários a API REST se mostrou mais eficiente do que a API GraphQL. Em um dos cenários, a API GraphQL respondeu a consulta na metade do tempo necessário para a API REST, sendo que em nenhum cenário a diferença foi menos do que 15\%. Os resultados da primeira hipótese podem ajudar um pouco a explicar por que os tempos de resposta da API GraphQL são menores do que os tempos da API REST. Uma vez que é preciso trafegar menos informações pela rede, as consultas são naturalmente respondidas mais rapidamente. Entretando, outro fator que influencia o tempo de resposta, é quanto tempo leva para processar as consultas.

A terceira hipótese afirmanva que tempo de utilização da CPU para as requisições REST. Os resultados mostraram que mesmo embor em um dos cenários a utilização da CPU foi práticamente igual entre as APIs, a medida que é necessário realizar consultas em objetos maiores, o tempo de utilização da CPU pela API REST aumenta signitivamente, alcançando uma diferença de mais de 40\% em relação a API GraphQL. 

Ambas as tecnicas oferecem uma soluções para o mesmo problema.

Diferença para a integração dos times frontend e backend

Clientes não precisam atualizar seu código para atender mudanças no servidor

\lipsum[31-33]
