% ----------------------------------------------------------
% Introdução (exemplo de capítulo sem numeração, mas presente no Sumário)
% ----------------------------------------------------------
\chapter[Introdução]{Introdução}
%\addcontentsline{toc}{chapter}{Introdução}
% ----------------------------------------------------------

Nas ultimas décadas, houve uma mudança para um modelo de computação chamado cliente/servidor, que aborda as falhas da computação centralizada em hosts. Claramente, o modelo de computação centralizado permanece válido em certos ambientes de negócios, no entanto, apesar de muitos benefícios, a computação centralizada é reconhecida como tendo promovido uma cultura de gerenciamento de informações que não conseguiu atender as necessidades de seus clientes.

Em 2010, ocorreu um grande avanço no número de APIs públicas impulsionado pela transição no modelo de comunicação entre aplicações distribuídas, onde estas passaram a utilizar amplamente o protocolo HTTP e o modelo cliente-servidor para a troca de informações na
World Wide Web \cite{tcc-ufsc}.

A adoção do REST (REpresentational State Transfer) como o método predominante para construir APIs públicas tem ofuscado qualquer outra tecnologia ou abordagem nos últimos anos. Embora várias alternativas (principalmente SOAP) ainda estejam presentes no mercado, adeptos do modelo SOA para construção de aplicações tomaram uma posição definitiva contra eles e optaram por REST como sua abordagem e JSON como seu formato de mensagem \cite{programmableweb-rest-losing}.

Segundo \citeonline{rest-webservice} REST é cada vez mais usado como alternativa ao “já antigo” SOAP em que a principal crítica a este é a burocracia, algo que o REST possui em uma escala muito menor. REST é baseado no \textit{design} do protocolo HTTP, que já possui diversos mecanismos embutidos para representar recursos como código de \textit{status}, representação de tipos de conteúdo, cabeçalhos, etc. O principal nesta arquitetura são as URLs do sistema e os \textit{resources} \footnote{resource é um recurso, entidade}. Ele aproveita os métodos HTTP para se comunicar.
